\title{Luciferase reporter assay meta-analysis}
\author{}
\date{\today}
\documentclass[12pt]{scrartcl}

\usepackage{amsmath}

\begin{document}

Let $X_{jk} \sim \mathcal{N}_{jk}$ where
$\mathcal{N}_{jk} = \mathcal{N}(\theta_k\mu_j,\theta_k^2\sigma_j)$
and $\theta_k$ is a random variable with $E[\theta_k] = 1$.
Let $x_{ijk}$ be the $i$'th normalized $F_{luc}/R_{luc}$ value of the $j$'th
configuration from the $k$'th batch, drawn from $X_{jk}$.

We wish to test whether or not $\mu_0 = \mu_1$. To do this, we will first
normalize out $\theta_k$ and then apply a t-test.

Since $E[\theta_k] = 1$ and $E[X_{jk}] = \theta_k\mu_j$, we can estimate
$\mu_j$ as $\bar{\mu}_j \approx \frac{1}{N_iN_k}\sum_{i,k}x_{ijk}$. Then we 
let $\hat{x}_{ijk} = \eta_k x_{ijk}$ where $\eta_k$
minimizes $\sum_{i,j} (\eta_k x_{ijk} - \bar{\mu}_j)^2$. Note that
$\eta_k \theta_k \approx 1$.
The values $\hat{x}_{ijk}$ then have distributon
$\eta_k \mathcal{N}_{jk} \approx \mathcal{N}(\mu_j, \sigma_j)$. This means
the hypothesis $\mu_0 = \mu_1$ can be tested using the populations
$\hat{x}_{i0k}$ and $\hat{x}_{i1k}$.

\end{document}
This is never printed
